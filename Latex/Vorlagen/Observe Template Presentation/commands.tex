\def\matlab{MATLAB }
\def\simulink{SIMULINK }
\def\colst#1{\color{blue}#1\color{black}} 
\definecolor{lightblue}{rgb}{0,0.5,0}
\def\colstd#1{\color{lightblue}#1\color{black}} 

\newcommand\trp{T} % transpose
\newcommand\ts{t_s}
\newcommand\Ts{\colst{T_s}}
\newcommand\dTs{\colst{\dot{T}_s}}
\newcommand\Tr{\colst{T_r}} 
\newcommand\dTr{\colst{\dot{T}_r}} 
\newcommand\Tb{\colst{T_b}} 
\newcommand\DTb{\Delta T_b}
\def\Tbr{T_{br}} 
\def\Ta{\colstd{T_a}}
\def\Tam{T_{a,m}}
\def\Td{\colst{T_{a1}}} 
\def\dTd{\colst{T_{a2}}} 
\def\Ty{\colst{T_{a3}}} 
\def\dTy{\colst{T_{a4}}}
\def\dotTd{\colst{\dot{T}_{a1}}} 
\def\dotdTd{\colst{\dot{T}_{a2}}} 
\def\dotTy{\colst{\dot{T}_{a3}}} 
\def\dotdTy{\colst{\dot{T}_{a4}}}
\newcommand\Xl{\colst{X_{l}}} 
\newcommand\Xu{\colst{X_{u}}} 
\newcommand\dV{\colstd{\dot{V}}} 
\newcommand\Kel{K}
\def\dVm{\dot{V}_m} 
\def\dVa{\dot{V}_a} 
\def\Vk{V_b}
\def\Vv{V_c}
\def\nst{\Phi}
\def\kr{k_b}
\def\khr{k_{r,b}}
\def\kra{k_{b,a}}
\def\Qab{\colstd{\dot{Q}_d}}
\def\Hl{\colst{H_l}}
\def\Hu{\colst{H_u}}
\def\Hll{75 \degree \mbox{C}}
\def\Hul{95 \degree \mbox{C}}
\def\Inp{\colst{U}}
\def\Pin{P_{max}\Inp}
\def\quant{\beta}
\def\mylabel{\label}
\def\tens{\texttt}
\def\xb{x}
\def\ub{u}
\def\monel{\mu}
\def\comment#1{\color{red}

\uppercase{#1}

\color{black}} 


\def\bmag#1#2{
\left(\begin{array}{*{#1}{c}}
#2
\end{array}
\right)
}
\def\unterlinie#1{\underline{#1}}

\newcommand{\bbma} {\begin{bmatrix} }
\newcommand{\ebma} {\end{bmatrix}}
\newcommand{\bpma} {\begin{pmatrix} }
\newcommand{\epma} {\end{pmatrix}}




 \def\vec#1{\ensuremath{\mathchoice
                     {\mbox{\boldmath$\displaystyle\mathbf{#1}$}}
                     {\mbox{\boldmath$\textstyle\mathbf{#1}$}}
                     {\mbox{\boldmath$\scriptstyle\mathbf{#1}$}}
                     {\mbox{\boldmath$\scriptscriptstyle\mathbf{#1}$}}}}%

% tensor
\def\tens#1{\relax\ifmmode\mathsf{#1}\else\textsf{#1}\fi}

\def\norm#1{\|#1\|} 

\def\utilde{\undertilde}

%\def\kruskal#1{\left[\left[\; #1 \;\right]\right]}

\newcommand{\tucker}[2]{\left[ #1 \right] \cdot #2}
\newcommand{\kruskalone}[1]{\left[ #1 \right] \cdot \tens{I}}
\newcommand{\kruskal}[2]{\left[ #1 \right] \cdot #2 }
\newcommand{\krusk}[1]{\left[ #1 \right] }
\newcommand{\kruskaltwo}[2]{\left[ #1 \right] \cdot #2}

%\newcommand{\cprod}[3]{\left\langle\!\!\!\left\langle \, #1 \,|\,  #2  \right.\right\rangle_{#3}}

%\newcommand{\cprod}[3]{\left\langle #1 \,,\,  #2  \right\rangle_{#3}}

\newcommand{\cprod}[3]{\left\langle\,#1\,\left|\,#2\,\right.\right\rangle_{#3}}
\newcommand{\aprod}[3]{\left\langle\,#1\,\left|\,#2\,\right.\right\rangle^+_{#3}}
\newcommand{\bprod}[3]{\left\langle\,#1\,\left|\,#2\,\right.\right\rangle^\Box_{#3}}
\newcommand{\hprod}[3]{\left\langle\,#1\,\left|\,#2\,\right.\right\rangle^\boxplus_{#3}}

%\newcommand{\cprod}[3]{\left\langle\,#1\,|\cdot |\,#2\,\right\rangle_{#3}}


%\newcommand{\hprod}[3]{\left\langle\!\left\langle #1 \,|\, #2  \right.\right\rangle^*_{#3}}

\newcommand{\iprod}[2]{\left\langle #1 \,,\,  #2  \right\rangle}


\newcommand{\btens}[1]{ 
\underline{\tens{#1}}}

\newcommand{\htens}[1]{ 
{\tens{#1}^*}}

\newcommand{\ctens}[1]{ 
\utilde{\tens{#1}}}

%\newcommand{\atens}[1]{ 
%\underline{\underline{\tens{#1}}}}

\newcommand{\atens}[1]{ 
\tens{#1}}

\newcommand{\cvec}[1]{%\raisebox{1mm}{
\utilde}
}}

%commands for scalars 
\newcommand{\cscal}[1]{%\raisebox{1mm}{
\utilde}
}}
\newcommand{\bscal}[1]{%\raisebox{1mm}{
\underline}
}}

\newcommand{\amat}[1]{ 
\underline{\underline{\mat{#1}}}}

%\newcommand{\amati}[1]{ 
%\underline{\underline{\mat{#1}\hspace{-1pt}}}\hspace{1pt}}

\newcommand{\amati}[1]{ 
\mat{#1}}


\newcommand{\cmat}[1]{%\raisebox{1mm}{
\utilde}
}}


\newcommand{\btensi}[1]{ 
\underline{\tens{#1}\hspace{-1pt}}\hspace{1pt}}


\newcommand{\ctensi}[1]{ 
\utilde{\tens{#1}\hspace{-1pt}}\hspace{1pt}}

\newcommand{\bvec}[1]{ 
\underline{\vec{#1}}}



\newcommand{\bveci}[1]{
\underline{\vec{#1}\hspace{-1pt}}\hspace{1pt}}



\newcommand{\tmat}[1]{\mbox{mat} \left( 
\mathbf{#1} \right)}

\newcommand{\tvec}[1]{\mbox{vec} \left( 
\mathbf{#1} \right)}

\newcommand{\hvec}[1]{ \mathbf{#1}}


%\newcommand{\fone}[0]{\vec{\nu}_1 }
%\newcommand{\fzero}[0]{\vec{\nu}_0 }
%\newcommand{\fdcare}[0]{\vec{\nu}_{-}}

%\newcommand{\fone}[0]{\boxminus_1 }

\newcommand{\fone}[0]{\framebox[4mm]{1}}
\newcommand{\fzero}[0]{\framebox[4mm]{0}}
%\newcommand{\fdcare}[0]{\framebox[4mm]{\phantom{$0$}\hspace{-5pt}$-$}}
\newcommand{\fdcare}[0]{\framebox[4mm]{\phantom{$0$}}}


\newcommand{\mat}[1]{
\mathbf{#1}}

\newcommand{\bmati}[1]{ 
\underline{\mathbf{#1}\hspace{-1pt}}\hspace{1pt}}

\newcommand{\bmat}[1]{ 
\underline{\mathbf{#1}}}


\newcommand{\bighadamard}
{ \bigotimes\hspace{-3.26ex}\bigoplus}

\newcommand{\hadamard}
{\circledast}
%{ \otimes\hspace{-2.03ex}\oplus\;}

\newcommand{\vecop}[1]
{ \mbox{vec}\hspace{1mm}({#1}) }

\newcommand{\mycitation}[2]
{
\vspace{-25mm}
\hfill
\begin{minipage}[t]{0.45\textwidth}
\textit{#1}
\end{minipage}
\vspace{5mm}

\hfill \textrm{#2}

\vspace{15mm}
}


\newcommand{\tikzfig}[4]
{
\begin{figure}[h!]
\begin{center}
\scalebox{#3}{
\input{../pic/#2/#1.tikz}
%Figure
}
\end{center}
\caption{#4}
\label{fig:#2--#1}
\end{figure}
}


\newcommand{\myfig}[4]
{
\begin{figure}[h!]
\begin{center}
\ifpdf
	\includegraphics[scale = #3]{../pic/#2/#1.pdf}
\else
	\includegraphics[scale = #3]{../pic/#2/#1.eps}
\fi
%Figure
\end{center}
\caption{#4}
\label{fig:#2--#1}
\end{figure}
}


\newcommand{\tvlone}[1]
{\mbox{$
\begin{array}{c}
\lceil 
\texttt{#1} 
\rfloor
\end{array}
$}}


\newcommand{\tvlnotone}[1]
{\mbox{$
\begin{array}{c}
\lceil 
\overline{\texttt{#1} }
\rfloor
\end{array}
$}}



\newcommand{\tvltwo}[2]
{\mbox{$
\begin{array}{c@{\hspace{0pt}}c@{\hspace{0pt}}c}
\lceil 
 &\texttt{#1} \\[-3pt]
 &\texttt{#2}
&\rfloor
\end{array}
$}}

\newcommand{\tvlthree}[3]
{\mbox{$
\begin{array}{c@{\hspace{0pt}}c@{\hspace{0pt}}c}
\lceil 
  &\texttt{#1} \\[-3pt]
  &\texttt{#2} \\[-3pt]
  &\texttt{#3}
&\rfloor
\end{array}
$}}

\newcommand{\tvlfour}[4]
{\mbox{$
\begin{array}{c@{\hspace{0pt}}c@{\hspace{0pt}}c}
\lceil 
  &\texttt{#1} \\[-3pt]
  &\texttt{#2} \\[-3pt]
  &\texttt{#3} \\[-3pt]
  &\texttt{#4}
&\rfloor
\end{array}
$}}


\newcommand{\booleanfunctwod}[8]
{
\begin{tikzpicture}[auto,node distance=2cm,thick]
\ifcase#1
  \node[state]  (A) {#5};
\or
  \node[state,fill=gray,draw=none,text=white] (A) {#5};
\fi
\ifcase#2
  \node[state] (B) [above of=A] {#6};
\or
  \node[state,fill=gray,draw=none,text=white] (B) [above of=A] {#6};
\fi
\ifcase#3
  \node[state] (C) [right of=A] {#7};
\or
  \node[state,fill=gray,draw=none,text=white] (C) [right of=A] {#7};
\fi
\ifcase#4
   \node[state] (D) [above of=C] {#8};
\or
  \node[state,fill=gray,draw=none,text=white] (D) [above of=C] {#8};
\fi
  \path (A) edge              node  {} (B)
        (A) edge              node  {} (C)
        (B) edge              node  {} (D)
        (C) edge              node  {} (D);
\end{tikzpicture}
}


%\newcommand{\booleanfuncthreed}[8]
%{
%\tikzstyle{state}=[circle,minimum size=2mm,draw=black!100]
%\begin{tikzpicture}[auto,node distance=10mm,thick]
%\ifcase#1
%  \node[state]  (A) {};
%\or
%  \node[state,fill=gray,draw=none,text=white] (A) {};
%\fi
%\ifcase#2
%  \node[state] (B) [right of=A] {};
%\or
%  \node[state,fill=gray,draw=none,text=white] (B) [right of=A] {};
%\fi
%\ifcase#3
%  \node[state] (C) [above of=A] {};
%\or
%  \node[state,fill=gray,draw=none,text=white] (C) [above of=A] {};
%\fi
%\ifcase#4
%   \node[state] (D) [above of=B] {};
%\or
%  \node[state,fill=gray,draw=none,text=white] (D) [above of=B] {};
%\fi
%\ifcase#5
%  \node[state]  (E) [above left of=A,node distance=6mm] {};
%\or
%  \node[state,fill=gray,draw=none,text=white,node distance=6mm] (E)  [above left of=A] {};
%\fi
%\ifcase#6
%  \node[state] (F) [right of=E] {};
%\or
%  \node[state,fill=gray,draw=none,text=white] (F) [right of=E] {};
%\fi
%\ifcase#7
%  \node[state] (G) [above of=E] {};
%\or
%  \node[state,fill=gray,draw=none,text=white] (G) [above of=E] {};
%\fi
%\ifcase#8
%   \node[state] (H) [above of=F] {};
%\or
%  \node[state,fill=gray,draw=none,text=white] (H) [above of=F] {};
%\fi
%
%
%  \path (A) edge              node  {} (B)
%        (A) edge              node  {} (C)
%        (A) edge              node  {} (E)
%        (B) edge              node  {} (D)
%        (B) edge              node  {} (F)
%        (C) edge              node  {} (D)
%        (C) edge              node  {} (G)
%        (D) edge              node  {} (H)
%        (E) edge              node  {} (F)
%        (E) edge              node  {} (G)
%        (F) edge              node  {} (H)
%        (G) edge              node  {} (H);
%\end{tikzpicture}
%}

\newcounter{ai}
\newcounter{ni}

\newcommand{\booleanfunctwodlist}[1]
{
\tikzstyle{state}=[circle,minimum size=2mm,draw=black!100,fill = gray]
\begin{tikzpicture}[auto,node distance=10mm,thick]
  \node[state] (A) {};
  \node[state] (B) [right of=A] {};
  \node[state] (C) [above of=A] {};
  \node[state] (D) [above of=B] {};
  \path (A) edge              node  {} (B)
        (A) edge              node  {} (C)
        (B) edge              node  {} (D)
        (C) edge              node  {} (D);
\setcounter{ai}{0}
\foreach \a in {#1} {
	\stepcounter{ai}
	\setcounter{ni}{0}
	\ifcase\a 
		\foreach \n in {A,B,C,D} {
			\stepcounter{ni}
			\ifnum\value{ni}=\value{ai}
  				\node[state,fill = white] at (node cs:name=\n) {};
				\breakforeach
			\fi
		}
	\fi
}
\end{tikzpicture}

}


\newcommand{\booleanfuncthreedlist}[1]
{
\tikzstyle{state}=[circle,minimum size=2mm,draw=black!100,fill = gray]
\begin{tikzpicture}[auto,node distance=10mm,thick]
  \node[state] (A) {};
  \node[state] (B) [right of=A] {};
  \node[state] (C) [above of=A] {};
  \node[state] (D) [above of=B] {};
  \node[state] (E) [above right of=A,node distance=6mm] {};
  \node[state] (F) [right of=E] {};
  \node[state] (G) [above of=E] {};
  \node[state] (H) [above of=F] {};
  \path (A) edge              node  {} (B)
        (A) edge              node  {} (C)
        (A) edge              node  {} (E)
        (B) edge              node  {} (D)
        (B) edge              node  {} (F)
        (C) edge              node  {} (D)
        (C) edge              node  {} (G)
        (D) edge              node  {} (H)
        (E) edge              node  {} (F)
        (E) edge              node  {} (G)
        (F) edge              node  {} (H)
        (G) edge              node  {} (H);
\setcounter{ai}{0}
\foreach \a in {#1} {
	\stepcounter{ai}
	\setcounter{ni}{0}
	\ifcase\a 
		\foreach \n in {A,B,C,D,E,F,G,H} {
			\stepcounter{ni}
			\ifnum\value{ni}=\value{ai}
  				\node[state,fill = white] at (node cs:name=\n) {};
				\breakforeach
			\fi
		}
	\fi
}
\end{tikzpicture}

}



\newcommand{\booleanfuncfourdlist}[1]
{
\tikzstyle{state}=[circle,minimum size=2mm,draw=black!100,fill = gray]
\begin{tikzpicture}[auto,node distance=18mm,thick]
  \node[state] (A) {};
  \node[state] (B) [right of=A] {};
  \node[state] (C) [above of=A] {};
  \node[state] (D) [above of=B] {};
  \node[state] (E) [above right of=A,node distance=10mm] {};
  \node[state] (F) [right of=E] {};
  \node[state] (G) [above of=E] {};
  \node[state] (H) [above of=F] {};
  \node[state] (I) [above left of=A,node distance=5mm] {};
  \node[state] (J) [right of=I] {};
  \node[state] (K) [above of=I] {};
  \node[state] (L) [above of=J] {};
  \node[state] (M) [above right of=I,node distance=10mm] {};
  \node[state] (N) [right of=M] {};
  \node[state] (O) [above of=M] {};
  \node[state] (P) [above of=N] {};

  \path (A) edge              node  {} (B)
        (A) edge              node  {} (C)
        (A) edge              node  {} (E)
        (A) edge              node  {} (I)
        (B) edge              node  {} (D)
        (B) edge              node  {} (F)
        (B) edge              node  {} (J)
        (C) edge              node  {} (D)
        (C) edge              node  {} (G)
        (C) edge              node  {} (K)
        (D) edge              node  {} (H)
        (D) edge              node  {} (L)
        (E) edge              node  {} (F)
        (E) edge              node  {} (G)
        (E) edge              node  {} (M)
        (F) edge              node  {} (H)
        (F) edge              node  {} (N)
        (G) edge              node  {} (H)
        (G) edge              node  {} (O)
        (H) edge              node  {} (P)
        (I) edge              node  {} (J)
        (I) edge              node  {} (K)
        (I) edge              node  {} (M)
        (J) edge              node  {} (L)
        (J) edge              node  {} (N)
        (K) edge              node  {} (L)
        (K) edge              node  {} (O)
        (L) edge              node  {} (P)
        (M) edge              node  {} (N)
        (M) edge              node  {} (O)
        (N) edge              node  {} (P)
        (O) edge              node  {} (P);
\setcounter{ai}{0}
\foreach \a in {#1} {
	\stepcounter{ai}
	\setcounter{ni}{0}
	\ifcase\a 
		\foreach \n in {A,B,C,D,E,F,G,H,I,J,K,L,M,N,O,P} {
			\stepcounter{ni}
			\ifnum\value{ni}=\value{ai}
  				\node[state,fill = white] at (node cs:name=\n) {};
				\breakforeach
			\fi
		}
	\fi
}
\end{tikzpicture}

}



%\begin{tikzpicture}[scale = 10]
%\coordinate (A) at (0.1,0.2);
%\coordinate (B) at (0.2,0.3);
%\coordinate (C) at (0.4,0.7);
%\coordinate (D) at (0.9,0.5);
%\small
%\node at (barycentric cs:A=0,B=1,C=1,D=0) {0110};
%\node at (barycentric cs:A=1,B=0,C=0,D=1) {1001};
%\node at (barycentric cs:A=0,B=0,C=0,D=0) {0000};
%\node at (barycentric cs:A=1,B=1,C=1,D=1) {1111};
%\end{tikzpicture}





\newcommand{\tvlgraphtwo}[6]{
\begin{tikzpicture}
	\pgfmathparse{#1} \let\xpz\pgfmathresult
	\pgfmathparse{#2} \let\xpo\pgfmathresult
	\pgfmathparse{#3} \let\ypz\pgfmathresult
	\pgfmathparse{#4} \let\ypo\pgfmathresult
	\pgfmathparse{#6} \let\scale\pgfmathresult

	\pgfmathparse{1-\xpz} \let\xnz\pgfmathresult
	\pgfmathparse{1-\xpo} \let\xno\pgfmathresult
	\pgfmathparse{1-\ypz} \let\ynz\pgfmathresult
	\pgfmathparse{1-\ypo} \let\yno\pgfmathresult

\ifcase#5
     \tikzstyle{every node}=[draw,circle,fill=white,minimum size=9mm,
                            inner sep=0pt]

	\node (A) at (0,0) {\tvlone{}};
	\node (B) at (\xpz*\ypz*\scale,\xpo*\ypo*\scale) {\tvlone{11}};
	\node (C) at (\xnz*\ypz*\scale,\xno*\ypo*\scale) {\tvlone{01}};
	\node (D) at (\xpz*\ynz*\scale,\xpo*\yno*\scale) {\tvlone{10}};
	\node (E) at (\xnz*\ynz*\scale,\xno*\yno*\scale) {\tvlone{00}};
   \node (F) at (\xpz*\scale,\xpo*\scale) {\tvlone{1-}}; 
   \node (G) at (\xnz*\scale,\xno*\scale) {\tvlone{0-}}; 
   \node (H) at (\ypz*\scale,\ypo*\scale) {\tvlone{-1}}; 
   \node (I) at (\ynz*\scale,\yno*\scale) {\tvlone{-0}}; 
   \node (J) at (\xnz*\ypz*\scale+\xpz*\ynz*\scale,\xno*\ypo*\scale+\xpo*\yno*\scale) {\tvltwo{01}{10}}; 
   \node (K) at (\xnz*\scale+\xpz*\ynz*\scale,\xno*\scale+\xpo*\yno*\scale) {\tvlnotone{11}}; 
   \node (L) at (\xnz*\scale+\xpz*\ypz*\scale,\xno*\scale+\xpo*\ypo*\scale) {\tvlnotone{10}}; 
   \node (M) at (\ypz*\scale+\xpz*\ynz*\scale,\ypo*\scale+\xpo*\yno*\scale) {\tvlnotone{00}}; 
   \node (N) at (\xnz*\ynz*\scale+\xpz*\ypz*\scale,\xno*\yno*\scale+\xpo*\ypo*\scale)  {\tvltwo{00}{11}}; 
   \node (O) at (\ynz*\scale+\xpz*\ypz*\scale,\yno*\scale+\xpo*\ypo*\scale) {\tvlnotone{01}}; 
   \node (P) at (\scale,\scale) {\tvlone{--}}; 
\or 
     \tikzstyle{every node}=[draw,circle,fill=blue,minimum size=1mm,
                            inner sep=0pt]

	\node (A) at (0,0) {};
	\node (B) at (\xpz*\ypz*\scale,\xpo*\ypo*\scale) {};
	\node (C) at (\xnz*\ypz*\scale,\xno*\ypo*\scale) {};
	\node (D) at (\xpz*\ynz*\scale,\xpo*\yno*\scale) {};
	\node (E) at (\xnz*\ynz*\scale,\xno*\yno*\scale) {};
   \node (F) at (\xpz*\scale,\xpo*\scale) {};
   \node (G) at (\xnz*\scale,\xno*\scale) {}; 
   \node (H) at (\ypz*\scale,\ypo*\scale) {};
   \node (I) at (\ynz*\scale,\yno*\scale){};
   \node (J) at (\xnz*\ypz*\scale+\xpz*\ynz*\scale,\xno*\ypo*\scale+\xpo*\yno*\scale) {};
   \node (K) at (\xnz*\scale+\xpz*\ynz*\scale,\xno*\scale+\xpo*\yno*\scale) {}; 
   \node (L) at (\xnz*\scale+\xpz*\ypz*\scale,\xno*\scale+\xpo*\ypo*\scale) {};
   \node (M) at (\ypz*\scale+\xpz*\ynz*\scale,\ypo*\scale+\xpo*\yno*\scale) {};
   \node (N) at (\xnz*\ynz*\scale+\xpz*\ypz*\scale,\xno*\yno*\scale+\xpo*\ypo*\scale) {};
   \node (O) at (\ynz*\scale+\xpz*\ypz*\scale,\yno*\scale+\xpo*\ypo*\scale) {};
   \node (P) at (\scale,\scale) {};

\fi


	\path (A) edge (B);
	\path	(A) edge (C);
	\path	(A) edge (D);
	\path	(A) edge (E);
	\path (B) edge (F);
	\path	(B) edge (H);
	\path	(B) edge (N);
	\path	(C) edge (H);
	\path	(C) edge (J);
	\path	(C) edge (G);
	\path	(D) edge (F);
	\path	(D) edge (I);
	\path	(D) edge (J);
	\path	(E) edge (I);
	\path	(E) edge (G);
	\path	(E) edge (N);
	\path	(F) edge (M);
	\path	(F) edge (O);
	\path	(G) edge (K);
	\path	(G) edge (L);
	\path	(H) edge (M);
	\path	(H) edge (L);
	\path	(I) edge (K);
	\path	(I) edge (O);
	\path	(J) edge (K);
	\path	(J) edge (M);
	\path	(K) edge (P);
	\path	(L) edge (N);
	\path	(L) edge (P);
	\path	(M) edge (P);
	\path	(N) edge (L);	
	\path	(N) edge (O);
	\path	(O) edge (P);

\end{tikzpicture}
 }



\newcommand{\zhegalkingraphtwo}[6]{
\begin{tikzpicture}
	\pgfmathparse{#1} \let\xpz\pgfmathresult
	\pgfmathparse{#2} \let\xpo\pgfmathresult
	\pgfmathparse{#3} \let\ypz\pgfmathresult
	\pgfmathparse{#4} \let\ypo\pgfmathresult
	\pgfmathparse{#6} \let\scale\pgfmathresult

	\pgfmathparse{1-\xpz} \let\xnz\pgfmathresult
	\pgfmathparse{1-\xpo} \let\xno\pgfmathresult
	\pgfmathparse{1-\ypz} \let\ynz\pgfmathresult
	\pgfmathparse{1-\ypo} \let\yno\pgfmathresult

\ifcase#5
     \tikzstyle{every node}=[draw,circle,fill=white,minimum size=15mm,
                            inner sep=0pt]

	\node (A) at (0,0) {$0$};
	\node (B) at (\xpz*\ypz*\scale,\xpo*\ypo*\scale) {$x_1 x_2$};
	\node (C) at (\xnz*\ypz*\scale,\xno*\ypo*\scale) {$\bar{x}_1 x_2$};
	\node (D) at (\xpz*\ynz*\scale,\xpo*\yno*\scale) {$x_1 \bar{x}_2$};
	\node (E) at (\xnz*\ynz*\scale,\xno*\yno*\scale) {$\bar{x}_1 \bar{x}_2$};
   \node (F) at (\xpz*\scale,\xpo*\scale) {$x_1 $}; 
   \node (G) at (\xnz*\scale,\xno*\scale) {$\bar{x}_1$}; 
   \node (H) at (\ypz*\scale,\ypo*\scale) {$x_2$}; 
   \node (I) at (\ynz*\scale,\yno*\scale) {$\bar{x}_2$}; 
   \node (J) at (\xnz*\ypz*\scale+\xpz*\ynz*\scale,\xno*\ypo*\scale+\xpo*\yno*\scale) {$\bar{x}_1 x_2 + x_1 \bar{x}_2$}; 
   \node (K) at (\xnz*\scale+\xpz*\ynz*\scale,\xno*\scale+\xpo*\yno*\scale) 
{$1- x_1 x_2$}; 
   \node (L) at (\xnz*\scale+\xpz*\ypz*\scale,\xno*\scale+\xpo*\ypo*\scale) 
{$1- x_1 \bar{x}_2$}; 
   \node (M) at (\ypz*\scale+\xpz*\ynz*\scale,\ypo*\scale+\xpo*\yno*\scale) {$1- \bar{x}_1 \bar{x}_2$}; 
   \node (N) at (\xnz*\ynz*\scale+\xpz*\ypz*\scale,\xno*\yno*\scale+\xpo*\ypo*\scale)  {$\bar{x}_1 \bar{x}_2 + x_1 x_2$}; 
   \node (O) at (\ynz*\scale+\xpz*\ypz*\scale,\yno*\scale+\xpo*\ypo*\scale) {$1-  \bar{x}_1 x_2$}; 
   \node (P) at (\scale,\scale) {$1$}; 
\or 
     \tikzstyle{every node}=[draw,circle,fill=blue,minimum size=1mm,
                            inner sep=0pt]

	\node (A) at (0,0) {};
	\node (B) at (\xpz*\ypz*\scale,\xpo*\ypo*\scale) {};
	\node (C) at (\xnz*\ypz*\scale,\xno*\ypo*\scale) {};
	\node (D) at (\xpz*\ynz*\scale,\xpo*\yno*\scale) {};
	\node (E) at (\xnz*\ynz*\scale,\xno*\yno*\scale) {};
   \node (F) at (\xpz*\scale,\xpo*\scale) {};
   \node (G) at (\xnz*\scale,\xno*\scale) {}; 
   \node (H) at (\ypz*\scale,\ypo*\scale) {};
   \node (I) at (\ynz*\scale,\yno*\scale){};
   \node (J) at (\xnz*\ypz*\scale+\xpz*\ynz*\scale,\xno*\ypo*\scale+\xpo*\yno*\scale) {};
   \node (K) at (\xnz*\scale+\xpz*\ynz*\scale,\xno*\scale+\xpo*\yno*\scale) {}; 
   \node (L) at (\xnz*\scale+\xpz*\ypz*\scale,\xno*\scale+\xpo*\ypo*\scale) {};
   \node (M) at (\ypz*\scale+\xpz*\ynz*\scale,\ypo*\scale+\xpo*\yno*\scale) {};
   \node (N) at (\xnz*\ynz*\scale+\xpz*\ypz*\scale,\xno*\yno*\scale+\xpo*\ypo*\scale) {};
   \node (O) at (\ynz*\scale+\xpz*\ypz*\scale,\yno*\scale+\xpo*\ypo*\scale) {};
   \node (P) at (\scale,\scale) {};

\fi


	\path (A) edge (B);
	\path	(A) edge (C);
	\path	(A) edge (D);
	\path	(A) edge (E);
	\path (B) edge (F);
	\path	(B) edge (H);
	\path	(B) edge (N);
	\path	(C) edge (H);
	\path	(C) edge (J);
	\path	(C) edge (G);
	\path	(D) edge (F);
	\path	(D) edge (I);
	\path	(D) edge (J);
	\path	(E) edge (I);
	\path	(E) edge (G);
	\path	(E) edge (N);
	\path	(F) edge (M);
	\path	(F) edge (O);
	\path	(G) edge (K);
	\path	(G) edge (L);
	\path	(H) edge (M);
	\path	(H) edge (L);
	\path	(I) edge (K);
	\path	(I) edge (O);
	\path	(J) edge (K);
	\path	(J) edge (M);
	\path	(K) edge (P);
	\path	(L) edge (N);
	\path	(L) edge (P);
	\path	(M) edge (P);
	\path	(N) edge (L);	
	\path	(N) edge (O);
	\path	(O) edge (P);

\end{tikzpicture}
 }


% The top side of a cube
%\newcommand\topside[4]{
%  \fill[fill=#4, draw=black,opacity=0.6,shift={(210:#1)},shift={(-30:#2)},
%  shift={(90:#3)}] (0,0) -- (30:0.7) -- (0,0.7) --(150:0.7)--(0,0);
%}
%
% The left side of a cube
%\newcommand\leftside[4]{
%  \fill[fill=#4, draw=black,opacity=0.8,shift={(210:#1)},shift={(-30:#2)},
%  shift={(90:#3)}] (0,0) -- (0,-0.7) -- (210:0.7) --(150:0.7)--(0,0);
%}
%
% The right side of a cube
%\newcommand\rightside[4]{
%  \fill[fill=#4, draw=black,opacity=1.0,shift={(210:#1)},shift={(-30:#2)},
%  shift={(90:#3)}] (0,0) -- (30:0.7) -- (-30:0.7) --(0,-0.7)--(0,0);
%}


%\newcommand\topside[7]{
%  \fill[fill=#4, draw=black,opacity=0.6,shift={(210:#1)},shift={(-30:#2)},
%  shift={(90:#3)}] (0,0) -- (30:#5) -- (0,#6) --(150:#7)--(0,0);
%}

% The left side of a cube
%\newcommand\leftside[7]{
%  \fill[fill=#4, draw=black,opacity=0.8,shift={(210:#1)},shift={(-30:#2)},
%  shift={(90:#3)}] (0,0) -- (0,-#5) -- (210:#6) --(150:#7)--(0,0);
%}
%
% The right side of a cube
%\newcommand\rightside[7]{
%  \fill[fill=#4, draw=black,opacity=1.0,shift={(210:#1)},shift={(-30:#2)},
%  shift={(90:#3)}] (0,0) -- (30:#5) -- (-30:#6) --(0,-#7)--(0,0);
%}


\newcommand\leftside[6]{
\pgfpathmoveto{\pgfpointxyz{#1}{#2}{#3}}
\pgfpathlineto{\pgfpointxyz{#1}{#2+#5}{#3}}
\pgfpathlineto{\pgfpointxyz{#1}{#2+#5}{#3+#6}}
\pgfpathlineto{\pgfpointxyz{#1}{#2}{#3+#6}}
\pgfpathlineto{\pgfpointxyz{#1}{#2}{#3}}
\pgfsetfillcolor{#4}
\pgfusepath{fill}
}


\newcommand\topside[6]{
\pgfpathmoveto{\pgfpointxyz{#1}{#2}{#3}}
\pgfpathlineto{\pgfpointxyz{#1+#5}{#2}{#3}}
\pgfpathlineto{\pgfpointxyz{#1+#5}{#2+#6}{#3}}
\pgfpathlineto{\pgfpointxyz{#1}{#2+#6}{#3}}
\pgfpathlineto{\pgfpointxyz{#1}{#2}{#3}}
\pgfsetfillcolor{#4}
\pgfusepath{fill}
}


\newcommand\rightside[6]{
\pgfpathmoveto{\pgfpointxyz{#1}{#2}{#3}}
\pgfpathlineto{\pgfpointxyz{#1+#5}{#2}{#3}}
\pgfpathlineto{\pgfpointxyz{#1+#5}{#2}{#3+#6}}
\pgfpathlineto{\pgfpointxyz{#1}{#2}{#3+#6}}
\pgfpathlineto{\pgfpointxyz{#1}{#2}{#3}}
\pgfsetfillcolor{#4}
\pgfusepath{fill}
}

% The cube 
\newcommand\cube[6]{
%\pgfsetxvec{\pgfpoint{7mm}{7mm}}
%\pgfsetyvec{\pgfpoint{-7mm}{7mm}}
%\pgfsetzvec{\pgfpoint{0}{-10mm}}

\pgfsetfillopacity{0.6}
  \topside{#1}{#2}{#3}{#4}{0.6}{0.6} 
\pgfsetfillopacity{1.0}
  \leftside{#1}{#2}{#3}{#5}{0.6}{0.6} 
\pgfsetfillopacity{0.8}
  \rightside{#1}{#2}{#3}{#6}{0.6}{0.6} 
}

\newcommand\cubevar[9]{
%\pgfsetxvec{\pgfpoint{7mm}{7mm}}
%\pgfsetyvec{\pgfpoint{-7mm}{7mm}}
%\pgfsetzvec{\pgfpoint{0}{-10mm}}
%\pgfsetfillopacity{0.6}
  \topside{#1}{#2}{#3}{#4}{#7}{#8} 
%\pgfsetfillopacity{1.0}
  \leftside{#1}{#2}{#3}{#5}{#8}{#9} 
%\pgfsetfillopacity{0.8}
  \rightside{#1}{#2}{#3}{#6}{#7}{#9} 
}



\newcommand\setanglesthreed{
\pgfsetxvec{\pgfpoint{8mm}{6mm}}
\pgfsetyvec{\pgfpoint{-6mm}{8mm}}
\pgfsetzvec{\pgfpoint{0}{-10mm}}
}

\newcommand\setanglesthreedalt{
\pgfsetxvec{\pgfpoint{8mm}{6mm}}
\pgfsetyvec{\pgfpoint{-6mm}{8mm}}
\pgfsetzvec{\pgfpoint{0}{10mm}}
}

\newcommand\setanglesthreedsquare{
\pgfsetxvec{\pgfpoint{10mm}{0mm}}
\pgfsetyvec{\pgfpoint{0mm}{10mm}}
\pgfsetzvec{\pgfpoint{-7mm}{-7mm}}
}

\newcommand\fiberz[7]{
\pgfsetfillopacity{0.8}
  \topside{#1}{#2}{#3}{#4}{0.6}{0.6} 
\pgfsetfillopacity{0.6}
  \leftside{#1}{#2}{#3}{#5}{0.6}{#7}
\pgfsetfillopacity{1.0}
  \rightside{#1}{#2}{#3}{#6}{0.6}{#7} 
}

\newcommand\fibery[7]{
\pgfsetfillopacity{0.6}
\topside{#1}{#2}{#3}{#4}{0.6}{#7} 
\pgfsetfillopacity{0.8}
\leftside{#1}{#2}{#3}{#5}{#7}{0.6} 
\pgfsetfillopacity{1.0}
\rightside{#1}{#2}{#3}{#6}{0.6}{0.6} 
}

\newcommand\fiberx[7]{
\pgfsetfillopacity{0.6}
\topside{#1}{#2}{#3}{#4}{#7}{0.6} 
\pgfsetfillopacity{0.8}
\leftside{#1}{#2}{#3}{#5}{0.6}{0.6} 
\pgfsetfillopacity{1.0}
\rightside{#1}{#2}{#3}{#6}{#7}{0.6} 
}


\newcommand\slicex[8]{
\pgfsetfillopacity{0.6}
  \topside{#1}{#2}{#3}{#4}{0.6}{#7} 
\pgfsetfillopacity{0.8}
  \leftside{#1}{#2}{#3}{#5}{#7}{#8}
\pgfsetfillopacity{1.0}
  \rightside{#1}{#2}{#3}{#6}{0.6}{#8} 
}

\newcommand\slicey[8]{
\pgfsetfillopacity{0.6}
  \topside{#1}{#2}{#3}{#4}{#7}{0.6}
\pgfsetfillopacity{1.0}
  \leftside{#1}{#2}{#3}{#5}{0.6}{#8}
\pgfsetfillopacity{0.8}
  \rightside{#1}{#2}{#3}{#6}{#7}{#8}
}


\newcommand\slicez[8]{
\pgfsetfillopacity{0.6}
  \topside{#1}{#2}{#3}{#4}{#7}{#8}
\pgfsetfillopacity{0.8}
  \leftside{#1}{#2}{#3}{#5}{#8}{0.6}
\pgfsetfillopacity{1.0}
  \rightside{#1}{#2}{#3}{#6}{#7}{0.6}
}



% Definition of \planepartition
% To draw the following plane partition, just write \planepartition{ {a, b, c}, {d,e} }.
%  a b c
%  d e

\newcounter{x}
\newcounter{y}
\newcounter{z}


\newcommand\planepartition[1]{
 \setcounter{x}{-1}
  \foreach \a in {#1} {
    \addtocounter{x}{1}
    \setcounter{y}{-1}
    \foreach \b in \a {
      \addtocounter{y}{1}
      \setcounter{z}{-1}
      \foreach \c in {1,...,\b} {
        \addtocounter{z}{1}
        \cube{\value{x}}{\value{y}}{\value{z}}{white}{gray}{black}
      }
    }
  }
}
\newcommand\tensorplotthree[4]{
 \setcounter{x}{-1}
  \foreach \a in {#1} {
    \addtocounter{x}{1}
    \setcounter{y}{-1}
    \foreach \b in \a {
      \addtocounter{y}{1}
      \setcounter{z}{-1}
      \foreach \c in {1,...,\b} {
        \addtocounter{z}{1}
        \cube{\value{x}}{\value{y}}{\value{z}}{#2}{#3}{#4}
      }
    }
  }
}



\tikzstyle{block} = [ draw, rectangle, % fill=blue!20, 
                      minimum height = 2em, 
                      minimum width = 2em ] %, node distance=15mm ]
\tikzstyle{sum} = [  draw, fill=gray!20, circle]
%\tikzstyle{fork} = [ fill, circle, maximum size=2pt ] 
\tikzstyle{fork} = [coordinate]%,node distance=10mm] 
\tikzstyle{input} = [coordinate]%,node distance=10mm]
\tikzstyle{output} = [coordinate]%,node distance=10mm]
\tikzstyle{pin} = [coordinate] % [pin edge={to-,thin,black}]

\newcommand\BDstandardcontrolloop[3]{

%\begin{tikzpicture}[node distance=15mm, auto]%, > = triangle 45 ]
\begin{tikzpicture}[node distance=12mm, auto, > =latex', thick ]
    % We need to set at bounding box first. Otherwise the diagram
    % will change position for each frame.
    \path[use as bounding box] (-1,0) rectangle (12,-2);
    \path[->]<1-> node[input] (reference) {};
    \path[->]<1-> node[block, right of=reference] (ff-controller) {#3}
                  (reference) edge node {$r$} (ff-controller);
    \path[->]<1-> node[sum, right of=ff-controller] (sume) {}
                  (ff-controller) edge node {} (sume);
    \path[->]<1-> node[block, right of=sume] (fb-controller) {#2}
                  (sume) edge node {$e$} (fb-controller);
    \path[->]<1-> node[sum, right of=fb-controller] (sumu) {}
                  (fb-controller) edge node {$u$} (sumu);
    \path[->]<1-> node[input, above of=sumu] (u) {load disturbance}
                  (u) edge node {$d$} (sumu);
    \path[->]<1-> node[block, right of=sumu] (plant) {#1}
                  (sumu) edge node {$v$} (plant);
    \path[->]<1-> node[sum, right of=plant] (sumy) {}
                  (plant) edge node {$x$} (sumy);
    \path[->]<1-> node[input, above of=sumy] (n) {}
                  (n) edge node {$n$} (sumy);
    \path[-]<1-> node[coordinate, right of=sumy,node distance=4mm]  (forky) {}
                  (sumy) edge node {} (forky);
    \path[->]<1-> node[output, right of=forky,node distance=8mm] (out) {$y$}
                  (forky) edge node {$y$} (out);
    \path[->, draw]<1-> (forky) -- +(0,-1) -| node[pos=0.8] {$-$} (sume);

\end{tikzpicture}
}


\newcommand\BDsystemwithsignals[3]{
\begin{tikzpicture}
\begin{axis}[
anchor=outer east,
ymin=0,ymax=2,
xmin=-0.1,xmax=2,
height=4cm,width=4cm,
enlargelimits=false,
%xtick=\empty,
%ytick=\empty,
xtick={0,1,2},
ytick={1,2},
%xlabel = $t$,
%ylabel = $u$,
axis on top=false]
\addplot[thick,blue]  
        plot[id=step,domain=-0.1:2,samples=52]
        function {#2};
\end{axis}
\begin{scope}[node distance=20mm, auto,> =latex',thick,anchor = east]
   \path[->]<1-> node[input] (input) {};
    \path[->]<1-> node[draw, rectangle, fill=blue!20, 
                      minimum height = 1cm, 
                      minimum width = 15mm , right of=input] (plant) {#1}
                  (input) edge node {} (plant);
    \path[->]<1-> node[output, right of=plant] (output) {}
                  (plant) edge node {} (output);
 \end{scope} 
 \path[->]<1-> node[coordinate,right of=output] (outplot) {} ;
\begin{axis}[ at={(output)},
anchor=outer west,
ymin=0,ymax=2,
xmin=-0.1,xmax=2,
height=4cm,width=4cm,
enlargelimits=false,
xtick={0,1,2},
ytick={1,2},
%xlabel = $t$,
%ylabel = $y$,
axis on top=false]
\addplot[thick,blue,smooth] 
        plot[id=step,domain=-0.1:2,samples=52]
        function {#3};
\end{axis} 
\end{tikzpicture}
}

\newcommand\BDsystemwithtext[3]{
\begin{tikzpicture}
  \draw node[anchor= east] (intext) {#2};
  \begin{scope}[node distance=20mm, auto,> =latex',thick,anchor = east]
     \path[->] node[input] (input) {};
     \path[->] node[draw, 
                      %fill=blue!20, 
                      rectangle, 
                      minimum height = 10mm, 
                      minimum width = 20mm , right of=input] (plant) {#1}
                  (input) edge node {} (plant);
     \path[->] node[output, right of=plant] (output) {}
                  (plant) edge node {} (output);
  \end{scope} 
  \draw node[anchor= west,at={(output)}] (outtext) {#3};
\end{tikzpicture}
}

\tikzstyle{lblock} = [ draw, rectangle, % fill=blue!20, 
                      minimum height = 10mm, 
                      minimum width = 25mm ] %, node distance=15mm ]

\newcommand\BDstandardhybridsystem[9]{

%\begin{tikzpicture}[node distance=15mm, auto]%, > = triangle 45 ]
\begin{tikzpicture}[node distance=25mm, auto, > =latex', thick]
    % We need to set at bounding box first. Otherwise the diagram
    % will change position for each frame.
    \path[use as bounding box] (-1,0) rectangle (12,-2);
    \node[input] (uc) {};
    \node[lblock, right of=uc, node distance=30mm] (sc) {#1};
    \node[lblock, below right of=sc] (qn) {#3};
    \node[lblock, below left of=qn] (sd) {#2};
    \node[lblock, above left of=sd] (in) {#4};
    \node[input,right of =sd, node distance=30mm] (ud) {#7};
    \node[output,right of =sc, node distance=30mm] (yc) {#6};
    \node[output,left of =sd, node distance=30mm] (yd) {#8};
    \draw[->, link] (sc.-10) -| node [pos=0.7, auto]  {#9} (qn.north);
    \draw[<-, link] (sc.-170) -| (in.north);
    \draw[->, link] (sd.170) -| (in.south);
    \draw[<-, link] (sd.10) -| (qn.south);
    \draw[->, link] (ud) -- node [pos=0.3, auto]  {#7} (sd.east);
    \draw[->, link] (uc) -- node [pos=0.3, auto]  {#5} (sc.west);
    \draw[<-, link] (yd) -- node [pos=0.3, auto, swap]  {#8} (sd.west);
    \draw[<-, link] (yc) -- node [pos=0.3, auto, swap]  {#6} (sc.east);

%
%    \path[->, draw] (qn.south) -- (sd.east)+(0,0.2) {};
%    \path[->, draw] (sd.west)+(0,0.2)  -| (in.south) {} ;
%    \path[->, draw] (in.north)  |- (sc.west)+(0,-0.2) {} ;
%    \path[->, draw]<1-> (sc.east) -- +(0,-0.2) -| node[pos=0.8] {} (q.north);
%    \path[->, draw]<1-> (sc.east) -- +(0,-0.2) -| node[pos=0.8] {} (q.north);
%     \node[input, right of = sd ] (ud) {};
%    \path[->, draw] (ud)+(0,-0.2) -- (sd.east) +(0,-0.2) {} ;
\end{tikzpicture}
}


\newcounter{parlevel}
\newcounter{serlevel}

%define a parallel block:  a tikz-pic with two nodes drawn on top of each other and some connecting lines
%the node contents comes from  #1 and #2
%the two nodes are named &#39;nTsp&#39;  and &#39;nBsp&#39;     where s and p are serial and parallel levels which increment
\newcommand{\parblock}[2]
{
   \tikz[baseline,remember picture,inner sep=0pt,outer sep=0pt,node distance=0.25cm]
   {
       \addtocounter{parlevel}{1}
       %special names for the two nodes:
       \def\nTsp{nT-\arabic{serlevel}-\arabic{parlevel}}
       \def\nBsp{nB-\arabic{serlevel}-\arabic{parlevel}}
       %
       %define the two nodes:
       \node(\nTsp){#1};
       \node[below=of \nTsp](\nBsp){#2};
       %
       %use bounding box so that the lines are based on the widest one
       \path (current bounding box.west) -- +(-0.125,0) coordinate (source);
       \path (current bounding box.east) -- +( 0.125,0) coordinate (dest);
       %
       %draw up/down and across from source (left)
       \draw (source) |- (\nTsp.west);
       \draw (source) |- (\nBsp.west);
       %
       %draw across and up/down to dest (right)
       \draw (\nTsp.east) -| (dest);
       \draw (\nBsp.east) -| (dest);
       %
       %add extra horizontal lines at source and dest
       \draw (source) -- +(-0.125,0);
       \draw (dest) -- +( 0.125,0);
       \addtocounter{parlevel}{-1}
   }
}


%\newcommand{\block}[1]
%{
%   \tikz[baseline,remember picture,inner sep=2pt]
%   {
%       \node[draw,shape=rectangle](#1){#1};
%   }
%}

%define a series block:  a tikz-pic with two nodes drawn next to each other and some connecting lines
%the node contents comes from  #1 and #2
%the two nodes are named &#39;nAsp&#39;  and &#39;nBsp&#39;     where x,y are serial and parallel levels which increment
\newcommand{\serblock}[2]
{
   \tikz[baseline,remember picture,inner sep=0pt,outer sep=0pt,node distance=0.25cm]
   {
       \addtocounter{serlevel}{1}
       %special names for the two nodes:
       \def\nLsp{nL-\arabic{serlevel}-\arabic{parlevel}}
       \def\nRsp{nR-\arabic{serlevel}-\arabic{parlevel}}
       %
       %define the two nodes:
       \node(\nLsp){#1};
       \node[right=of \nLsp](\nRsp){#2};
       %
       %define source and dest just past the nAsp and nBsp nodes
       \path (\nLsp.west) -- +(-0.125,0) coordinate (source);
       \path (\nRsp.east) -- +( 0.125,0) coordinate (dest);
       %
       %add lines at extreme ends
       \draw (source) -- (\nLsp.west);
       \draw (\nLsp.east) -- (\nRsp.west);
       \draw (\nRsp.east) -- (dest);
       \addtocounter{serlevel}{-1}
   }
}




\definecolor{darkblue}{rgb}{0,0,.4}
\definecolor{darkred}{rgb}{.4,0,0}
\definecolor{darkgreen}{rgb}{0,.4,0}




